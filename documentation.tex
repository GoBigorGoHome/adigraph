\documentclass{report}

\title{Adigraph, V1.2}
\author{Luca Cappelletti}
\date{March 2018}

\usepackage{adigraph}
\usepackage{xcolor}
\usepackage{minted}
\usepackage{multicol}
\usepackage{graphicx} % for images and generally graphics
\usepackage{caption} % enabling of nice captions
\usepackage{subcaption} % and subcaptions of images
\definecolor{mintedbackground}{rgb}{0.95,0.95,0.95}
\setminted{
  bgcolor=mintedbackground,
  fontfamily=tt,
  linenos=true,
  numberblanklines=true,
  numbersep=5pt,
  gobble=0,
  frame=leftline,
  framerule=0.4pt,
  framesep=2mm,
  funcnamehighlighting=true,
  tabsize=4,
  obeytabs=false,
  mathescape=false
  samepage=false, %with this setting you can force the list to appear on the same page
  showspaces=false,
  showtabs =false,
  texcl=false,
}

%%%%%%%%%%%%%%%%%%%%%%%%%%%%%%%%%%%%%%%%%%%%%%%%%%%%%%%%%
% THE FOLLOWING CENTERS ALL FLOATING ITEMS BY DEFAULT   %
%%%%%%%%%%%%%%%%%%%%%%%%%%%%%%%%%%%%%%%%%%%%%%%%%%%%%%%%%

\makeatletter
\g@addto@macro\@floatboxreset\centering
\makeatother

\makeatletter
\apptocmd\subcaption@minipage{\centering}{}{}
\makeatother

\makeatletter
\providecommand*\setfloatlocations[2]{\@namedef{fps@#1}{#2}}
\makeatother
\setfloatlocations{figure}{H}
\setfloatlocations{table}{H}

\begin{document}

\maketitle

\chapter{Introduction}
\textbf{Adigraph} is a latex library for drawing directed graphs and augmenting directed graphs, and to draw cuts over them.

It handles automatically the positioning of labels, with the exception of the horizontal position, and the inclinations of cuts.

\section{Setup}
\subsection{Installing the dependencies}
Clearly you need to have texlive installed. Then, make sure you have the following packages:

\begin{description}
	\item[fp] Used for floating point calculations.
	\item[xparse] Used for elaborating parameters.
	\item[xstring] Used for operations on strings.
	\item[etoolbox] Used for operations on lists.
	\item[tikz] Used for drawing the actual graphs.
	\item[tikz calc library] Used for some internal calculations in tikz.
\end{description}

To be sure you can run the following, that will install the packages only if they are not already present:

\begin{minted}{sh}
sudo tlmgr install etoolbox xstring fp xparse tikz
\end{minted}

\subsection{Installing Adigraph}
You can install Adigraph, if it isn't already present in your setup, by running the following on Unix systems:

\begin{minted}{sh}
sudo tlmgr install adigraph
\end{minted}

On windows you should check on your package manager of choice (some latex distribution have a tlmgr implementation on windows too.)

\chapter{Usage}
\section{Creating a new graph}
Here we create a new Adigraph object, called \textit{myAdigraph}.

\begin{minted}{latex}
\NewAdigraph{myAdigraph}{
	<nodes here, separated by semicolon>
}{
	<edges here, separated by semicolon>
}{
	<cuts here, separated by semicolon>
}
\end{minted}

\section{Adding nodes}
We set its nodes with the following syntax: \textit{<node name: \(x\) coordinate, \(y\) coordinate, color: label>}.

\begin{figure}
	\begin{subfigure}{0.49\textwidth}
		\begin{minted}{latex}
\NewAdigraph{myAdigraph}{
 	s:0,0;
 	t:4,0;
}
\myAdigraph{}
\end{minted}
	\end{subfigure}
	\begin{subfigure}{0.49\textwidth}
		\NewAdigraph{myAdigraph}{
			s:0,0;
			t:4,0;
		}
		\myAdigraph{}
	\end{subfigure}
\end{figure}


\subsection{Custom node colors}
To color a node you can use the following syntax: \textit{<node name: \(x\) coordinate, \(y\) coordinate, textual color>}. For example, to draw s in red and t in blue we would write:

\begin{figure}
	\begin{subfigure}{0.49\textwidth}
		\begin{minted}{latex}
\NewAdigraph{myAdigraph}{
 	s:0,0,red;
 	t:4,0,blue;
}
\myAdigraph{}
\end{minted}
	\end{subfigure}
	\begin{subfigure}{0.49\textwidth}
		\NewAdigraph{myAdigraph}{
			s:0,0,red;
			t:4,0,blue;
		}
		\myAdigraph{}
	\end{subfigure}
\end{figure}

Tested available colors are: red, blue, black, green. You may extend the possible colors with LaTex libraries such as xcolor.

\subsection{Custom node labels}
To add a custom label you can use the following syntax: either \textit{<node name: \(x\) coordinate, \(y\) coordinate: node label>} or \textit{<node name: \(x\) coordinate, \(y\) coordinate, textual color: node label>} will work:

\begin{figure}
	\begin{subfigure}{0.49\textwidth}
		\begin{minted}{latex}
\NewAdigraph{myAdigraph}{
 	s:0,0,red:start;
 	t:4,0:end;
}
\myAdigraph{}
\end{minted}
	\end{subfigure}
	\begin{subfigure}{0.49\textwidth}
		\NewAdigraph{myAdigraph}{
			s:0,0,red:start;
			t:4,0:end;
		}
		\myAdigraph{}
	\end{subfigure}
\end{figure}

\section{Adding edges}
We set its nodes with the following syntax: \textit{<node name: \(x\) coordinate, \(y\) coordinate, color : label>}.

\subsection{A simple edge}
\begin{figure}
	\begin{subfigure}{0.49\textwidth}
		\begin{minted}{latex}
\NewAdigraph{myAdigraph}{
 	s:0,0;
 	t:4,0;
}{
	s,t;
}
\myAdigraph{}
\end{minted}
	\end{subfigure}
	\begin{subfigure}{0.49\textwidth}
		\NewAdigraph{myAdigraph}{
			s:0,0;
			t:4,0;
		}{
			s,t;
		}
		\myAdigraph{}
	\end{subfigure}
\end{figure}

\subsection{A colored simple edge}
\begin{figure}
	\begin{subfigure}{0.49\textwidth}
		\begin{minted}{latex}
\NewAdigraph{myAdigraph}{
 	s:0,0;
 	t:4,0;
}{
	s,t,red;
}
\myAdigraph{}
\end{minted}
	\end{subfigure}
	\begin{subfigure}{0.49\textwidth}
		\NewAdigraph{myAdigraph}{
			s:0,0;
			t:4,0;
		}{
			s,t,red;
		}
		\myAdigraph{}
	\end{subfigure}
\end{figure}

\subsection{A weighted edge}
\begin{figure}
	\begin{subfigure}{0.49\textwidth}
		\begin{minted}{latex}
\NewAdigraph{myAdigraph}{
 	s:0,0;
 	t:4,0;
}{
	s,t:56;
}
\myAdigraph{}
\end{minted}
	\end{subfigure}
	\begin{subfigure}{0.49\textwidth}
		\NewAdigraph{myAdigraph}{
			s:0,0;
			t:4,0;
		}{
			s,t:56;
		}
		\myAdigraph{}
	\end{subfigure}
\end{figure}

\subsection{A weighted edge with label}
\begin{figure}
	\begin{subfigure}{0.49\textwidth}
		\begin{minted}{latex}
\NewAdigraph{myAdigraph}{
 	s:0,0;
 	t:4,0;
}{
	s,t:56:myLabel;
}
\myAdigraph{}
\end{minted}
	\end{subfigure}
	\begin{subfigure}{0.49\textwidth}
		\NewAdigraph{myAdigraph}{
			s:0,0;
			t:4,0;
		}{
			s,t:56:myLabel;
		}
		\myAdigraph{}
	\end{subfigure}
\end{figure}

\subsection{Edge in both directions}
\begin{figure}
	\begin{subfigure}{0.49\textwidth}
		\begin{minted}{latex}
\NewAdigraph{myAdigraph}{
 	s:0,0;
 	t:4,0;
}{
	s,t;
	t,s;
}
\myAdigraph{}
\end{minted}
	\end{subfigure}
	\begin{subfigure}{0.49\textwidth}
		\NewAdigraph{myAdigraph}{
			s:0,0;
			t:4,0;
		}{
			s,t;
			t,s;
		}
		\myAdigraph{}
	\end{subfigure}
\end{figure}

\subsection{Edge with weights in both directions}
\begin{figure}
	\begin{subfigure}{0.49\textwidth}
		\begin{minted}{latex}
\NewAdigraph{myAdigraph}{
 	s:0,0;
 	t:4,0;
}{
	s,t:5;
	t,s:5;
}
\myAdigraph{}
\end{minted}
	\end{subfigure}
	\begin{subfigure}{0.49\textwidth}
		\NewAdigraph{myAdigraph}{
			s:0,0;
			t:4,0;
		}{
			s,t:5;
			t,s:5;
		}
		\myAdigraph{}
	\end{subfigure}
\end{figure}

\subsection{Positioning labels}
\begin{figure}
	\begin{subfigure}{0.49\textwidth}
		\begin{minted}{latex}
\NewAdigraph{myAdigraph}{
	1:0,0;
	2:0,2;
	3:4,2;
	4:4,0;
}{
	1,3,red:1:a:near start;
	2,4,blue:1:b:near end;
}
\myAdigraph{}
\end{minted}
	\end{subfigure}
	\begin{subfigure}{0.49\textwidth}
		\NewAdigraph{myAdigraph}{
			1:0,0;
			2:0,2;
			3:4,2;
			4:4,0;
		}{
			1,3,red:1:a:near start;
			2,4,blue:1:b:near end;
		}
		\myAdigraph{}
	\end{subfigure}
\end{figure}

\subsection{Positioning weights}
\begin{figure}
	\begin{subfigure}{0.49\textwidth}
		\begin{minted}{latex}
\NewAdigraph{myAdigraph}{
	1:0,0;
	2:0,2;
	3:4,2;
	4:4,0;
}{
	1,3,red:1::near start;
	2,4,blue:1::near end;
}
\myAdigraph{}
\end{minted}
	\end{subfigure}
	\begin{subfigure}{0.49\textwidth}
		\NewAdigraph{myAdigraph}{
			1:0,0;
			2:0,2;
			3:4,2;
			4:4,0;
		}{
			1,3,red:1::near start;
			2,4,blue:1::near end;
		}
		\myAdigraph{}
	\end{subfigure}
\end{figure}


\subsection{Multiple edges with weights}
\begin{figure}
	\begin{subfigure}{0.49\textwidth}
		\begin{minted}{latex}
\NewAdigraph{myAdigraph}{
	s:0,0;
	t:4,0;
	1:0,3;
	2:4,3;
}{
	s,t:5;
	t,s:5;
	s,1:5;
	1,s:5;
	1,2:5;
	2,1:5;
	2,t:5;
	t,2:5;
	t,1:5;
	1,t:5;
}
\myAdigraph{}
\end{minted}
	\end{subfigure}
	\begin{subfigure}{0.49\textwidth}
		\NewAdigraph{myAdigraph}{
			s:0,0;
			t:4,0;
			1:0,3;
			2:4,3;
		}{
			s,t:5;
			t,s:5;
			s,1:5;
			1,s:5;
			1,2:5;
			2,1:5;
			2,t:5;
			t,2:5;
			t,1:5;
			1,t:5;
		}
		\myAdigraph{}
	\end{subfigure}
\end{figure}

\section{Augmenting paths}
An augmenting path is specified by the following syntax: \textit{<comma separated list of nodes:units>}. It is \textbf{very important} to note that incremental paths called upon the same object are memorized by default.

\NewAdigraph{myAdigraph}{
	s:0,0;
	1:2,2;
	3:2,-2;
	2:6,2;
	4:6,-2;
	t:8,0;
}{
	s,1:25;
	s,3:25;
	3,4:25;
	1,2:35;
	2,t:20;
	4,t:30;
	3,1:10;
	4,2:10;
	2,3:15::near start;
	4,1:5::near start;
}

\begin{figure}
	\begin{subfigure}{0.49\textwidth}
		\begin{minted}{latex}
\NewAdigraph{myAdigraph}{
	s:0,0;
	1:2,2;
	3:2,-2;
	2:6,2;
	4:6,-2;
	t:8,0;
}{
	s,1:25;
	s,3:25;
	3,4:25;
	1,2:35;
	2,t:20;
	3,1:10;
	4,2:10;
	2,3:15::near start;
	4,1:5::near start;
}
\myAdigraph{
	s,3,4,2,t:5;
}
\end{minted}
	\end{subfigure}
	\begin{subfigure}{0.49\textwidth}
		\myAdigraph{
			s,3,4,2,t:5;
		}
	\end{subfigure}
\end{figure}

For example, suppose now we'd like to send another 5 units on the graph edited by the previous incremental path, we'll have just to write the following:

\begin{figure}
	\begin{subfigure}{0.49\textwidth}
		\begin{minted}{latex}
\myAdigraph{
	s,3,4,1,2,t:5;
}
\end{minted}
	\end{subfigure}
	\begin{subfigure}{0.49\textwidth}
		\myAdigraph{
			s,3,4,1,2,t:5;
		}
	\end{subfigure}
\end{figure}

\begin{figure}
	\begin{subfigure}{0.49\textwidth}
		\begin{minted}{latex}
\myAdigraph{
	s,3,4,t:15;
}
\end{minted}
	\end{subfigure}
	\begin{subfigure}{0.49\textwidth}
		\myAdigraph{
			s,3,4,t:15;
		}
	\end{subfigure}
\end{figure}

\begin{figure}
	\begin{subfigure}{0.49\textwidth}
		\begin{minted}{latex}
\myAdigraph{
	s,1,4,t:5;
}
\end{minted}
	\end{subfigure}
	\begin{subfigure}{0.49\textwidth}
		\myAdigraph{
			s,1,4,t:5;
		}
	\end{subfigure}
\end{figure}

\begin{figure}
	\begin{subfigure}{0.49\textwidth}
		\begin{minted}{latex}
\myAdigraph{
	s,1,2,t:10;
}
\end{minted}
	\end{subfigure}
	\begin{subfigure}{0.49\textwidth}
		\myAdigraph{
			s,1,2,t:10;
		}
	\end{subfigure}
\end{figure}

\begin{figure}
	\begin{subfigure}{0.49\textwidth}
		\begin{minted}{latex}
\myAdigraph{
	s,1,2,4,t:5;
}
\end{minted}
	\end{subfigure}
	\begin{subfigure}{0.49\textwidth}
		\myAdigraph{
			s,1,2,4,t:5;
		}
	\end{subfigure}
\end{figure}

\section{Cuts}

The following is to add cuts to show minimum cuts for example, the syntax is: \textit{<first node, second node;>}

\begin{figure}
	\begin{subfigure}{0.49\textwidth}
		\begin{minted}{latex}
\myAdigraph{}{
	3,4;
	2,t;
}
\end{minted}
	\end{subfigure}
	\begin{subfigure}{0.49\textwidth}
		\myAdigraph{}{
			2,t;
			3,4;
		}
	\end{subfigure}
\end{figure}

\subsection{Colored cuts}
If you'd like to color the cuts you just have to add the color as follows: \textit{<first node, second node, color;>}

\begin{figure}
	\begin{subfigure}{0.49\textwidth}
		\begin{minted}{latex}
\myAdigraph{}{
	3,4,red;
	2,t,blue;
}
\end{minted}
	\end{subfigure}
	\begin{subfigure}{0.49\textwidth}
		\myAdigraph{}{
			2,t,red;
			3,4,blue;
		}
	\end{subfigure}
\end{figure}

\chapter{Warnings}
\section{Reserved words}
I reserve to use for the package the following tokens:

\begin{multicols}{2}
	\begin{enumerate}
		\item Adigraph
		\item AdigraphBuildEdge
		\item AdigraphBuildEdgeWrapper
		\item AdigraphBuildNode
		\item AdigraphBuildPath
		\item AdigraphCalculateOrientation
		\item AdigraphCountPaths
		\item AdigraphCurrentPathNumber
		\item AdigraphCutBuilder
		\item AdigraphDrawEdge
		\item AdigraphDrawNode
		\item AdigraphEdgeBuilder
		\item AdigraphEdgeDrawer
		\item AdigraphEdgeName
		\item AdigraphElaboratePath
		\item AdigraphElaboratePath
		\item AdigraphExecuteCutBuilder
		\item AdigraphFirstNode
		\item AdigraphMemorizeEdge
		\item AdigraphMemorizeNode
		\item AdigraphNodeBuilder
		\item AdigraphNodeName
		\item AdigraphNumberOfPaths
		\item AdigraphPathBuilder
		\item AdigraphProcessCuts
		\item AdigraphProcessEdges
		\item AdigraphProcessNodes
		\item AdigraphProcessPaths
		\item AdigraphSecondNode
		\item AdigraphSimpleSum
		\item AdigraphTwinEdgeWeight
		\item NewAdigraph
	\end{enumerate}
\end{multicols}

\end{document}